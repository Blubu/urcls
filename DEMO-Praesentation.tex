% Copyright 2014-2015 by Marei Peischl
%
% This file may be distributed and/or modified\\
% under the LaTeX Project Public License
%
%Version 1.1 (2015/06/15)

\documentclass[aspectratio=169,ngerman,ph]{URbeamer} %Sprach & Farbauswahl, entweder mit Kürzel einzelner Einrichtungen, oder faculties für alle Fakultäten oder all, für sämtliche Farben des Farbschemas
\usepackage[utf8]{inputenc}
\usepackage[T1]{fontenc}
\usepackage{babel}
%Für die Hausschriftart der Universität Regensburg, falls installiert:
\usepackage{frutigernext}

\usepackage{tabularx}

%Definitionen für die Titlepage
\title{\LaTeX{}-beamer}
\institute{Fakultät für Physik}
\subtitle{Im CD der Universität Regensburg}
\author[Marei Peischl]{Marei Peischl \href{mailto:tex@mareipeischl.de}{tex@ mareipeischl.de}}

\begin{document}
\frame[plain]{\titlepage}

\begin{frame}{Hausschriftart: frutigernext}
	Frutigernext kann innerhalb des Universitätsnetzes der Uni Regensburtg auf der Website des \LaTeX{}-Kurses heruntergeladen werden: \url{http://www.physik.uni-regensburg.de/studium/edverg/latex/files/cd/cd.phtml}
\end{frame}

\begin{frame}{Auswahl der teilhabenden Fakultäten}
Die Fakultätsfarben werden durch Angabe der zugehörigen Dokumentenklassenoption ausgewählt, folgende Möglichkeiten existieren:
\begin{tabularx}{\linewidth}{lX}
lov&Leitung, Organe, Verwaltung\\
ffg&Chancengleicheit und Familie\\
asz&Service-Einrichtungen für Studierende\\
rw&Fakultät für Rechtswissenschaft\\
ww&Fakultät für Wirtschaftswissenschaften\\
kt&Fakultät für katholosche Theologie\\
pkgg&Fakultät für Philosophie, Kunst-, Geschichts- und Gesellschaftswissenschaften\\
pps&Fakultät für Psychologie, Pädagogik und Sportwissenschaft\\
slk&Fakultät für Sprach-,Literatur- und Kulturwissenschaften\\
\end{tabularx}
\end{frame}

\begin{frame}{Fortsetzung der Liste aller Einrichtungsoptionen}
\begin{tabularx}{\linewidth}{lX}
bvm&Fakultät für Biologie und vorklinische Medizin\\
mat&Fakultät für Mathematik\\
ph&Fakultät für Physik\\
chp&Fakultät für Chemie und Pharmazie\\
med&Fakultät für Medizin\\
ub&Universitätsbibliothek\\
zsk&Zentrum für Sprache und Kommunikation\\
eur&Europaeum (Ost-West-Zentrum)\\
zhw&Zentrum für Hochschul- und Wissenschaftsdidaktik\\
rul&Regensburg Universitätszentrum für Lehrerbildum\\
zfw&Zentrum für Weiterbildung\\
spo&Sportzentrum \\
rz&Rechenzentrum\\
all&alle Einrichtungen\\
faculties&alle Fakultäten\\
\end{tabularx}
\end{frame}
\begin{frame}{Entwurfsmodus}
\texttt{draft}-Option ersetzt den farbigen Streifen auf der Titelseite sowie die Kopfzeile durch leere schwarze Rechtecke.
Zudem die üblichen Änderungen der Dokumentenklasse beamer durchgeführt.
\end{frame}
\begin{frame}{Oftmal zweimaliges Kompilieren nötig}
Für die korrekte Darstellung der farbigen Streifen und die korrekte Positionierung der Schriftzüge ist oftmals zweimaliges Kompilieren nötig.
\end{frame}

\begin{frame}{TikZ-externalize-Funktion}
	Die Klassenoption externalize unterstützt die TikZ-Funktion \glqq{}externalize\grqq. Dazu muss URbeamer mit der entsprechenden Option geladen werden und anschließend die Ausgabe der pdf-Dateien mit \texttt{\textbackslash{}tiktexternalize} aktiviert werden. Für die Nutzung dieser Funktion muss \texttt{pdflatex} mit  der Option \texttt{-shell-escape} ausgeführt werden.
	
\end{frame}

\end{document}